\documentclass{article}
\usepackage{amsmath}
\usepackage{amsfonts}
\usepackage{amssymb}
\usepackage{amsthm}

\usepackage{float}
\title{CS 517 \\ Homework 6}
\author{Ameyassh Nagarajan}
\date{June 2024}


\begin{document}
\maketitle
\section{Problem 1}

\subsection*{RP Definition}
A language \(L\) is in \(\text{RP}\) (Randomized Polynomial time) if there exists a probabilistic polynomial-time Turing machine \(M\) such that:
\begin{itemize}
    \item If \(x \in L\), \(M(x)\) accepts with probability at least \(\frac{1}{2}\).
    \item If \(x \notin L\), \(M(x)\) always rejects.
\end{itemize}

\subsection*{P/poly Definition}
A language \(L\) is in \(\text{P/poly}\) if there exists a language \(A\) in \(\text{P}\) and a set of advice strings \(\{a_0, a_1, a_2, \ldots\}\) such that:
\begin{itemize}
    \item \(|a_n| \leq n^{O(1)}\), where \(a_n\) is the advice string for inputs of length \(n\).
    \item For all \(x\) of length \(n\), \(x \in L\) if and only if \((x, a_n) \in A\).
\end{itemize}

Alternatively, a language \(L\) is in \(\text{P/poly}\) if there exists a family of circuits \(\{C_0, C_1, \ldots\}\) such that:
\begin{itemize}
    \item \(|C_n| \leq n^{O(1)}\).
    \item For all \(x\) of length \(n\), \(x \in L\) if and only if \(C_n(x) = 1\).
\end{itemize}

\subsection{Proof}
Let \(L\) be a language in \(\text{RP}\). By definition, there exists a probabilistic polynomial-time Turing machine \(M\) such that:
\begin{itemize}
    \item If \(x \in L\), \(M(x)\) accepts with probability at least \(\frac{1}{2}\).
    \item If \(x \notin L\), \(M(x)\) always rejects.
\end{itemize}

For each input length \(n\), consider the probabilistic machine \(M\). By the probabilistic nature of \(M\), there exists a fixed random tape \(r_n\) of polynomial length such that \(M(x, r_n)\) works correctly for at least half of the inputs of length \(n\). Specifically, \(M(x, r_n)\) accepts \(x \in L\) with probability at least \(1/2\) and always rejects \(x \notin L\).

Construct a circuit \(C_n\) that simulates \(M(x, r_n)\) for inputs of length \(n\). This circuit \(C_n\) is polynomial in size because it simulates a polynomial-time Turing machine with a fixed random tape. The size of \(C_n\) is bounded by some polynomial \(p(n)\).

To handle the remaining inputs not correctly decided by \(C_n\), we recursively apply the same process. For the remaining half of the inputs, there exists another random tape \(r'_n\) that works for at least half of those inputs. We construct a new circuit \(C'_n\) for these inputs. Repeating this process, we eventually cover all inputs of length \(n\).

Thus, for each input length \(n\), we construct a family of circuits \(\{C_n\}\) such that for every input \(x\) of length \(n\), there is a circuit in the family that correctly decides whether \(x \in L\).

Each random tape \(r_n\) used in the construction is of polynomial length in \(n\), satisfying the requirement that the advice length is polynomially bounded by the input length.

Hence, \(L \in \text{P/poly}\), and we have shown that \(\text{RP} \subseteq \text{P/poly}\).


    

\subsection*{Conclusion}
Since we have shown that for any language in \(\text{RP}\), there exists a family of polynomial-size circuits that correctly decides the language, we conclude that:
\[
\text{RP} \subseteq \text{P/poly}
\]
%%%%%%%%%%%%%%%%%%%%%%%%%%%%%%%%%%%%%%%%%%%%%%%%%%%%%%%%%%%%%%%%%%%%%%%%%%
\pagebreak
\section{Problem 2}
\section*{EXP is a Subset of XPP}

To show that \(\text{EXP} \subseteq \text{XPP}\), we need to construct a probabilistic Turing machine \(M\) for any language \(L \in \text{EXP}\) such that:
\begin{itemize}
    \item[1] \(x \in L \implies \Pr[M(x) = 1] \geq 1/2\)
    \item[2] \(x \not\in L \implies \Pr[M(x) = 1] < 1/2\)
\end{itemize}



with expected polynomial running time.

\subsection*{Construction of the Probabilistic Turing Machine}

Let \(L \in \text{EXP}\) be decided by a deterministic Turing machine \(D\) that runs in time \(2^{p(n)}\) for some polynomial \(p(n)\). We construct a probabilistic Turing machine \(M\) that, on input \(x\) of length \(n\):
\begin{itemize}
    \item[1] With probability \(1 - \frac{1}{2^n}\), \(M\) performs a trivial polynomial-time computation (e.g., returns 0).
    \item[2] With probability \(\frac{1}{2^n}\), \(M\) simulates \(D\) on \(x\), which takes time \(2^{p(n)}\).
\end{itemize}



\subsection*{Expected Running Time}

Let \(T_{\text{poly}}(x)\) be the polynomial-time computation and \(T_{\text{exp}}(x)\) be the exponential-time computation. The expected running time \(E[T(x)]\) is:
\[
E[T(x)] = \left(1 - \frac{1}{2^n}\right) T_{\text{poly}}(x) + \frac{1}{2^n} T_{\text{exp}}(x)
\]
Given that \(T_{\text{poly}}(x)\) is polynomial and \(T_{\text{exp}}(x) = 2^{p(n)}\), we have:
\[
E[T(x)] = \left(1 - \frac{1}{2^n}\right) \text{poly}(n) + \frac{1}{2^n} 2^{p(n)}
\]
Since \(\frac{1}{2^n} 2^{p(n)} = 2^{p(n) - n}\) and \(p(n)\) is a polynomial, \(2^{p(n) - n}\) becomes negligible for large \(n\). Thus, \(E[T(x)]\) remains polynomial.

\subsection*{Probability Analysis Using Chernoff Bound}

Define \(X_i\) as a Bernoulli random variable which is 1 if the \(i\)-th trial runs in exponential time, and 0 otherwise. Let \(X = \sum_{i=1}^n X_i\) be the sum of \(n\) trials. The expected value \(E[X]\) is:
\[
E[X] = n \cdot \frac{1}{2^n}
\]

Applying the Chernoff bound, for \(\delta > 0\):
\[
\Pr[X \geq (1+\delta)E[X]] \leq \exp\left(-\frac{\delta^2 E[X]}{2 + \delta}\right)
\]

Choosing \(\delta = \frac{1}{2}\), we get:
\[
\Pr\left[X \geq \frac{3}{2}E[X]\right] \leq \exp\left(-\frac{\left(\frac{1}{2}\right)^2 E[X]}{2 + \frac{1}{2}}\right) = \exp\left(-\frac{E[X]}{10}\right)
\]
Since \(E[X] = n \cdot \frac{1}{2^n}\), we have:
\[
\Pr\left[X \geq \frac{3}{2}E[X]\right] \leq \exp\left(-\frac{n}{10 \cdot 2^n}\right)
\]
This probability is extremely small for large \(n\).


\subsection*{Conclusion for EXP}

By carefully balancing the probabilities and running times, we ensure the expected running time remains polynomial while the correctness conditions for \(x \in L\) and \(x \not\in L\) are satisfied. This approach effectively "sneaks" in the exponential computation with sufficiently low probability, thereby proving \(\text{EXP} \subseteq \text{XPP}\).


\section*{Pushing Beyond EXP}

To push the method beyond \(\text{EXP}\), we consider classes such as \(\text{2-EXP}\), the class of languages decidable by deterministic Turing machines in doubly exponential time, \(2^{2^{p(n)}}\) for some polynomial \(p(n)\).

\subsection*{Construction of the Probabilistic Turing Machine}

Let \(L \in \text{2-EXP}\) be decided by a deterministic Turing machine \(D\) that runs in time \(2^{2^{p(n)}}\) for some polynomial \(p(n)\). We construct a probabilistic Turing machine \(M\) that, on input \(x\) of length \(n\):
\begin{itemize}
    \item[1] With probability \(1 - \frac{1}{2^{2^n}}\), \(M\) performs a trivial polynomial-time computation.
    \item[2] With probability \(\frac{1}{2^{2^n}}\), \(M\) simulates \(D\) on \(x\), which takes time \(2^{2^{p(n)}}\).
\end{itemize}

\subsection*{Expected Running Time}

Let \(T_{\text{poly}}(x)\) be the polynomial-time computation and \(T_{\text{exp}}(x)\) be the doubly exponential-time computation. The expected running time \(E[T(x)]\) is:
\[
E[T(x)] = \left(1 - \frac{1}{2^{2^n}}\right) T_{\text{poly}}(x) + \frac{1}{2^{2^n}} T_{\text{exp}}(x)
\]
Given that \(T_{\text{poly}}(x)\) is polynomial and \(T_{\text{exp}}(x) = 2^{2^{p(n)}}\), we have:
\[
E[T(x)] = \left(1 - \frac{1}{2^{2^n}}\right) \text{poly}(n) + \frac{1}{2^{2^n}} 2^{2^{p(n)}}
\]
Since \(\frac{1}{2^{2^n}} 2^{2^{p(n)}} = 2^{2^{p(n)} - 2^n}\) and \(2^{p(n)} - 2^n\) is extremely negative for large \(n\), this term becomes negligible. Thus, \(E[T(x)]\) remains polynomial.

\subsection*{Probability Analysis Using Chernoff Bound}

Define \(X_i\) as a Bernoulli random variable which is 1 if the \(i\)-th trial runs in exponential time, and 0 otherwise. Let \(X = \sum_{i=1}^n X_i\) be the sum of \(n\) trials. The expected value \(E[X]\) is:
\[
E[X] = n \cdot \frac{1}{2^{2^n}}
\]

Applying the Chernoff bound, for \(\delta > 0\):
\[
\Pr[X \geq (1+\delta)E[X]] \leq \exp\left(-\frac{\delta^2 E[X]}{2 + \delta}\right)
\]

To illustrate the effect of increasing \(\delta\), we calculate the bound for different values of \(\delta\):

1. For \(\delta = \frac{1}{2}\):
   \[
   \Pr[X \geq \frac{3}{2}E[X]] \leq \exp\left(-\frac{\left(\frac{1}{2}\right)^2 E[X]}{2 + \frac{1}{2}}\right) = \exp\left(-\frac{E[X]}{10}\right)
   \]

2. For \(\delta = \frac{3}{2}\):
   \[
   \Pr\left[X \geq \left(1 + \frac{3}{2}\right) E[X]\right] \leq \exp\left(-\frac{\left(\frac{3}{2}\right)^2 E[X]}{2 + \frac{3}{2}}\right) = \exp\left(-\frac{9 E[X]}{14}\right)
   \]

3. For \(\delta = 2\):
   \[
   \Pr[X \geq 3E[X]] \leq \exp\left(-\frac{4E[X]}{4}\right) = \exp\left(-E[X]\right)
   \]

In general, as \(\delta\) increases, the term \(\frac{\delta^2 E[X]}{2 + \delta}\) increases, making \(\exp\left(-\frac{\delta^2 E[X]}{2 + \delta}\right)\) smaller. This means the probability of \(X\) deviating from \(E[X]\) by a factor of \(1 + \delta\) decreases exponentially.


\subsection*{Conclusion for Pushing Beyond EXP}

By carefully balancing the probabilities and running times, we ensure the expected running time remains polynomial while the correctness conditions for $x \in L$ and $x \not\in L$ are satisfied. This approach effectively "sneaks" in the exponential computation with sufficiently low probability, thereby proving \(\text{EXP} \subseteq \text{XPP}\) and extending the method to higher complexity classes like \(\text{2-EXP}\).


\end{document}
